\chapter{Endboss}
Der Kampf mit dem Endboss, welcher auf der obersten Plattform des Levels zu finden ist, wurde speziell f"ur den Bosskampf gescriptet.
Er erscheint "uber eine Colliderbox, die beim betreten den Spawn des Bosses startet. Nach Ablauf eines kurzen Timers beginnt dieser zu erscheinen.\newline

Im Kampf selbst wurden "ahnliche Logikabfragen wie bei der normalen Gegner KI implementiert, jedoch in einer spezielleren Form, da sich durch gew"unschte Bossf"ahigkeiten diese nicht mit dem normalen KI Script umsetzen lassen.

Der Boss hat 3 Angriffe die von bestimmte Bedingungen haben:
\begin{description}[align=left]
	\item[\textbf{Schlag mit linker Hand}]
	Wenn sich der Spieler in der Trefferzone der linken Hand befindet\newline
	\item[\textbf{Schlag mit rechter Hand}]
	 Wenn sich der Spieler in der Trefferzone der rechten befindet\newline
	\item[\textbf{Schreien}]
	 Keine Bedingung. L"asst den Spieler f"ur eine gewisse Zeit bewegungsunf"ahig werden.
\end{description}

Damit diese drei Angriffe nicht etwa in statischer Reihenfolge ausgef"uhrt werden, und etwas mehr Dynamik in den Kampfablauf bringen, wurde hier ein Zufallsalgorithmus implementiert.
Dieser entscheidet zuf"allig welcher der m"oglichen Angriffe gew"ahlt werden soll, unter Ber"ucksichtigung der letzten ausgef"uhrten Aktionen.
Eine der m"oglichen Aktionen wird proportional zu ihren vergangenen Aktionsphasen, seit der letzten Verwendung immer wachsen.
Beim ersten Angriff besteht also f"ur jede Funktion eine Chance von $\frac{1}{3}$.
Wird dann beispielsweise der Schlag mit der linken Hand ausgef"uhrt, ist die Chance f"ur nochmals diese Aktion $\frac{1}{5}$, f"ur den Schrei und den Schlag mit der rechten Hand jeweils $\frac{2}{5}$.\newline

Besonderheit hierbei ist, dass die Schl"age zweimal hintereinander ausgef"uhrt werden, der Z"ahler wird lediglich auf 1 zur"uckgesetzt, w"ahrend es ausgeschlossen ist, dass zweimal ein Schrei ausgef"uhrt wird.

Beispielsweise der Code f"ur eine der gew"ahlten Aktionen und f"ur die Berechnung der Wahrscheinlichkeiten.

\begin{lstlisting}[breaklines=true]
void RightHandHit()
{
	//Variablen fuer den Code, welche attacke benutzt werden soll
	timesSinceScream++;
	timesSinceLeftHand++;
	timesSinceRightHand = 1;
	
	//Variablen für die Animation
	idleStateExecuted = false;
	anim.SetTrigger("RightHandAttack");
}


void getProbabilities()
{
	int options = timesSinceScream;
	if (inLeftHitBox)
	{ 
		options += timesSinceLeftHand;
	}
	if(inRightHitBox)
	{
		options += timesSinceRightHand;
	}
	screamProb = (float)timesSinceScream / (float)options;
	
	leftHandProb = inLeftHitBox ? (float)timesSinceLeftHand / (float)options : 0.0f;
	rightHandProb = inRightHitBox ? (float)timesSinceRightHand / (float)options : 0.0f;
}
\end{lstlisting}

Die Entscheidungen des Bosskampfes in einem Behaviour Tree dargestellt k"onnte folgenderma"sen aussehen:
\begin{figure}
	\centering
	\includegraphics[height=9cm]{images/behaviour_boss.png}
	\caption{Verhalten des Endboss' als Behaviour Tree}
	\label{fig:behaviour_tree_boss}
\end{figure}


Ist der Boss besiegt, f"allt ein kleines Easter Egg vom Himmel, welches eine Dialogbox mit Text an sich gebunden hat, und sich relativ zum Objekt mit bewegt.